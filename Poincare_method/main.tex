\documentclass[bachelor, och, coursework, times]{SCWorks}

\usepackage[T2A]{fontenc}
\usepackage[utf8]{inputenc}
\usepackage[english,russian]{babel}
\usepackage{geometry}

%\geometry{verbose,a4paper,tmargin=2cm,bmargin=2cm,lmargin=2.5cm,rmargin=1.5cm}

\begin{document}

%\tableofcontents

1 Название метода
2 пошаговый алгоритм действия (не масштабно, выявить суть в 4 шагах или типа того)
3 формулировка задачи (преаисываешь условие)
4 применение метода (выкладки при решеинии)


$u+u+\epsilon u^3=0$


методика линдштедта-пуанкаре

\section{Описание метода}

Пусть у нас есть уравнение Дюффинга вида $\frac{d^2x^*}{(dt^*)^2}+f(x^*)=0$, где $f$ - нелинейная функция от $x^*$, $\frac{d^2x^*}{(dt^*)^2}$ - определённое ускорение системы, $f(x^*)$ - нелинейная восстанавливающая сила.

 Вместо переменной t вводится новая переменная$t=\tau(1+\epsilon\omega_1+\epsilon^2\omega_2+\ldots)$, где $\omega$ есть некоторая постоянная величина, определяемая только значениями параметра $\epsilon$ 

Мы будем использовать специальный вид уравнения Дюффинга:
$$\frac{d^2u^*}{(dt^*)^2}+k_1u^*+k_3(u^*)^3=0 $$

где $k_1 0 $, а $k_3$ любое.

Приведём наше уравнение к безразмерному виду. С этой целью выберем некоторые характерные масштабы задачи — линейный $U*$ и временной $Т*$ — и положим
$$ u = \frac{u^*}{U}, \; \; \; t = \frac{t^*}{T} $$


Выразив дифференциалы \frac{d}{dt^*} \frac{d^2}{(dt^*)^2} через новые переменные исходное уравнение перепишется в виде

u..+k_1(T^*)^2u+k_3(T^*)^2(U^*)^2u^3=0

блаблабла

и тогда уравнение перепишется как $$u..+u+\epsilon u^3=0$$.

блаблабла

Начальные условия $u(0)=x_0, \; \; \; u.(0)=x._0$.

блаблабла

\newpage

Для того чтобы выявить зависимость частоты от степени нелинейности системы, введем частоты w непосредственно в дифференциальное уравнение (). 
Сделаем замену $\tau=\omega t$, где w есть некоторая постоянная величина, определяемая только значениями параметра е.

Далее выразим дифференциалы через новую переменную и получим
$\frac{d}{dt} = \omega \frac{d}{d\tau} , \; \frac{d^2}{dt^2} = \omega^2 \frac{d^2}{d\tau^2}$ и таким образом исходное уравнение (1) перепишется в виде : $\omega^2u^{''}+u+\epsilon u^3=0$, где производная берётся по $\tau$

Чтобы определить значения переменных $u$ и $\omega$.

Будем искать их в виде разложений по степеням $\epsilon$, т. е. положим
$ u = ....$ , $\omega=1+\epsilon \omega_1+\ldots$


При этом первый член разложения для со представляет собой 
частоту линейной системы, которая в нашем случае равна единице.
Далее, исходя из требования, что разложение для функции и
должно быть равномерным при всех т, в процессе вычислений
можно определить все последующие поправки к частоте линейной
системы.
:
Так, подстановка (4.68) и (4.69) в (4.67). дает после всех преобразований:

$$u_0^{''}+u_0+\epsilon(u_1^{''}+u_1+u_0^3+2\omega_1 u_0^{''}+\ldots=0$$.

Приравнивая нулю коэффициенты при $\epsilon^{0}$ и $\epsilon^{1}$ получаем

$u_0^{''}+u_=0, u_1^{''}+u_1=-u_0^3-2\omega_1u_0^{''}$

Общее решение первого даётся формулой $u_0=\alpha cos(\tau+\beta)$, где $\alpha$, $\beta$ некоторые произвольные постоянные.





%положим $u=\sum\limits_{n=0}^\infty \epsilon^n u_n.$

%приравняв коэффициенты при равных степенях $\epsilon$, получим уравнения %для последовательного определения $u_m$. Решения для $u_m$ не содержат %вековых членов только при опреленных значениях $\omega_m$.




\end{document}
