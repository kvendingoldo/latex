\documentclass[bachelor, och, coursework, times]{SCWorks}

\usepackage[T2A]{fontenc}
\usepackage[utf8]{inputenc}
\usepackage[english,russian]{babel}
\usepackage{geometry}

%\geometry{verbose,a4paper,tmargin=2cm,bmargin=2cm,lmargin=2.5cm,rmargin=1.5cm}

\begin{document}

\bigskip\bigskip\bigskip

\hfill

\hfill

\hfill

\hfill

\hfill

\hfill

\hfill

\hfill

\hfill

\begin{center}
\textbf{\Largeметодика Линдштедта-Пуанкаре}
\\
Шаров А.В.
\medskip
\end{center}

\tableofcontents

\intro

Метод прямымого разложения по параметру даёт приближенное решение задачи $\ddot{u}+u+\epsilon u^3=0\;,$ при малых, но конечных $\epsilon$, однако это разложение не будет равномерно пригодным для больших значений $t$.

Нарушение пригодности прямого разложения связано с тем, что с его помощью
невозможно описать зависимость частоты системы от степени
ее нелинейности. Таким образом, можно заранее утверждать, что
любое разложение, не учитывающее эту зависимость, окажется
несостоятельным. В настоящее время, однако, разработаны целые
ряды методов, позволяющих получать для подобного рода задач
равномерно пригодные разложения.

Решение нашей задачи является функцией независимой 
переменной $t$ и параметра $\epsilon$, а частота системы $\omega$ (её мы рассмотрим позже) является
функцией параметра $\epsilon$ (т. е. степени нелинейности системы).

Это обстоятельство используется для построения равномерно пригодного решения с помощью разложения функций $u$ и $\omega$ в ряд по степеням параметра $\epsilon$ — в этом состоит так называемая методика Линдштедта—Пуанкаре. 

\section{Уравнение Дюффинга}

Свободные колебания консервативных систем с одной степенью свободы во многих случаях описываются уравнением:

$$ \frac{d^2x^*}{dt^{*2}}+f(x^*)=0, \eqno(1)$$ 
где $f$ - некоторая нелинейная функция от $x^*$; при этом член $\frac{d^2x^*}{dt^{*2}}$ определяет ускорение системы, а $f(x^*)$ - нелинейную восстанавливающую силу.
Координата $x^*=x^*_0$ определяет положение равновесия системы; Отсюда ясно, что $f(x^*_0)=0$.
Предположим, что $f$ является аналитической функцией в в точке $x^*=x^*_0$; тогда в окрестности этой точки её можно разложить в ряд Тейлора:

$$f(x^*)=k_1(x^*-x_0)+k_2(x^*-x_0)^2+\ldots \;,$$
где $$k_n=\frac{1}{n!}\frac{d^nf}{dx^{*n}}x_0^*.$$
Таким образом, уравнение (1) можно переписать в виде:

$$ \frac{d^2x^*}{dt^{*2}}+k_1(x^*-x_0)+k_2(x^*-x_0)^2+\ldots=0 \eqno(2)$$
Уравнение (2) описывает движение данной системы вблизи положения равновесия. Вводя преобразование $u^*=x^*-x^*_0\;$ из уравнения (2) получим:

$$ \frac{d^2x^*}{dt^{*2}}+k_1u^*+k_2u^{*2}+\ldots=0.$$
Это общий вид уравнения описывающего свободные колебания консервативных систем с одной степенью свободы. Нам интересен частный вид этого уравнения:

$$ \frac{d^2x^*}{dt^{*2}}+k_1u^*+k_3u^{*3}=0\;, \eqno(3)$$
где $k_1 > 0$, а $k_3$ может быть любым.
Уравнение (3) и называется уравнением Дюффинга. 

В заключение, приведём уравнение Дюффинга к безразмерному виду. Выберем характерные масштабы задачи - линейный $U^*$ и временной $T^*$ - и положим:

$$ u=\frac{u^*}{U^*}, \; \; \; t=\frac{t^*}{T^*}.$$
По правилам дифференциирования:

$$ \frac{d}{dt^*}=\frac{d}{dt}\frac{dt}{dt^*}=\frac{1}{T^*}\frac{d}{dt} $$

$$ \frac{d^2}{dt^{*2}}=\frac{1}{T^{*2}}\frac{d^2}{dt^2} $$
При этом уравнение (3) преобразуется к виду:

$$\ddot{u}+k_1T^{*2}u+k_3T^{*2}U^{*2}u^3=0 \eqno(4)$$
выберем $T^*$ такое, что $k_1T^{*2}=1$ и положим $\epsilon=k_3T^{*2}U^{*2}=\frac{k_3U^{*2}}{k_1}$.
Теперь перепишем уравнение (4):

$$\ddot{u}+u+\epsilon u^3=0\;,$$
где $\epsilon$ - безразмерная величина характеризующая степень нелинейности системы.
Именно на примере уравнения (4) мы и будем изучать метод Линдштедта-Пуанкаре.
\newpage


\section{Описание метода}



Будем искать решения автономного уравнения $\ddot{x}+\omega^2x=\epsilon f(x,\dot{x})$.

\begin{enumerate} 

\item 
Введём замену $\tau=\omega t$ и посчитаем новые дифференциалы: 
$$\frac{d}{dt}=\omega\frac{d}{dt}$$
$$\frac{d^2}{dt^2}=\omega^2\frac{d^2}{dt^2}$$
Таким образом в наше уравнение теперь входит переменная $\omega$. Ищем $x,\omega$ в виде разложений по степенням $\epsilon$. Положим: 
$$x=x_0(t)+\epsilon u_1(t)+\ldots$$
$$\omega=1+\epsilon \omega_1+\ldots$$

\item
Подставляем $x,\omega$ в исходное уравнение, приравнивая кооэфициенты при $\epsilon_0, \epsilon_1$ к нулю и решаем получившиеся уравнения.

\item
Избавляемся от секулярных членов в решенях промежуточных уравнений с помощью подбора парамеров $\omega_1, \omega_2, \ldots$ так, чтобы наш секулярный член при соответсвующем члене был равен нулю.

Альтернативный способ: мы можем избавиться от секулярных членов исходя из вида неоднородности в правой части получившегося уравнения.

\item
Подставим промежуточные решения в наши исходные разложения $x,\omega$ и получим решения.

\end{enumerate}

\section{Применение метода}

Пусть у нас есть безразмерное уравнение Дюффинга: $$\ddot{u}+u+\epsilon u^3=0. \eqno(1)$$
Для начала сделаем замену $\tau=\omega t$, где $\omega$ есть некоторая постоянная величина, определяемая только значениями параметра $\epsilon$.
\\
Переходя от аргумента $t$ к овой переменной $\tau$ и используя правило дифференциирования сложной функции, получим:
$$\frac{d}{dt} = \omega \frac{d}{d\tau} , \;\;\; \frac{d^2}{dt^2} = \omega^2 \frac{d^2}{d\tau^2}$$ 
и таким образом исходное уравнение (1) перепишется в виде : 
$$\omega^2u^{''}+u+\epsilon u^3=0, \eqno(2)$$ 
где штрихи означают дифференцирование по переменной $\tau$.
\\
Чтобы определить значения переменных $u$ и $\omega$ будем искать их в виде разложений по степеням $\epsilon$, т. е. положим
$$ u = u_0(\tau)+\epsilon u_1(\tau)+\cdots \;, \eqno(3)$$
$$\omega=1+\epsilon \omega_1+\ldots \;. \eqno(4)$$
При этом первый член разложения для $\omega$ представляет собой 
частоту линейной системы, которая в нашем случае равна единице.
\\
Подставим (3) и (4) в (2) и получим:
$$(1+\epsilon \omega_1+\ldots)^2(u_0(\tau)+\epsilon u_1(\tau)+\cdots)^{''}+(u_0(\tau)+\epsilon u_1(\tau)+\cdots)+\epsilon (u_0(\tau)+\epsilon u_1(\tau)+\cdots)^3=0.$$
Отсюда:
$$u_0^{''}+u_0+\epsilon (u_1^{''}+u_1+u_0^3+2\omega_1u_0^{''})+\ldots=0$$
Приравняем кооэфициенты при $\epsilon_0$ и $\epsilon_1$ к нулю:
$$u_0^{''}+u_0=0 \eqno(5)$$
$$u_1^{''}+u_1=-u_0^3-2\omega_1u_0^{''} \eqno(6) $$
Общее решение (5) даётся формулой: 
$$u_0=\alpha\cos{(\tau+\beta)} \;,$$
где $\alpha$, $\beta$ некоторые произвольные постоянные.
\\
Теперь перепишем (6) в виде: 
$$u_1^{''}+u_1=-\alpha^3\cos^3{(\tau+\beta)}+2\omega_1\alpha\cos{(\tau+\beta)}.$$
Отсюда следует: 
$$u_1^{''}+u_1=(2\omega_1\alpha-\frac{3}{4}\alpha^3)\cos{(\tau+\beta)}-\frac{1}{4}\alpha^3\cos{(3\tau+3\beta)}. \eqno(7)$$
Частное решение уравнения (7) можно представить в виде:
$$u_1=\frac{1}{2}(2\omega_1\alpha-\frac{3}{4}\alpha^3)\tau\sin{(\tau+\beta)}+\frac{1}{32}\alpha^3\cos{(3\tau+3\beta)}$$
Видно, что данное решение для уравнение (7) содержит секулярный член, а следовательно решение неравномерно, а мы не должны допустить секулярных членов в выражениях для $u_1, u_2, \cdots$ и тд.
\\
В данном случае мы можем подобрать параметр $\omega_1$ таким образом, чтобы исключить соотвествеющий секулярный член. С этой целью положим кооэфициент при секулярном члене равным нулю, т.е. примем:
$$2\omega_1\alpha-\frac{3}{4}\alpha^3=0; \eqno(8)$$
Теперь, учитывая (8) уравнение (7) примет вид: 
$$u_1=\frac{1}{32}\alpha^3\cos{(3\tau+3\beta)}. \eqno(9)$$
Если исключить тривиальный случай $\alpha=0$, то условие (8) удоволетворяется при 
$$\omega_1=\frac{3}{8}\alpha^2. \eqno(10)$$

Для получения уравнения (8) позволяющего исключить секулярный член из выражения для $u_1$ нам нет необходимости строить в явном виде соответсвенное частное разложение(как показано выше).
Вместо этого мы можем исходя из вида неоднородности в правой части (7), которая собственно и определения характер функции $u_1$, просто положить равной нулю коэфициент при $\cos{(\tau+\beta)}$, поскольку именно это слагаемое ответственно за появление секулярного члена при $u_1$.

Подстановка (5) и (9) в (3) даёт:

$$u=\alpha\cos{(\tau+\beta)}+\frac{1}{32}\epsilon\alpha^2\cos{(3\tau+3\beta})+\ldots \eqno(11)$$
Подставим $\omega_1$ из (10) в (4) получим:
$$\omega=1+\frac{3}{8}\epsilon\alpha^2+\ldots \eqno(12)$$
Так как $\tau=t\omega$, то (5) можно переписать  в виде:
$$u=\alpha\cos{[(1+\frac{3}{8}\epsilon\alpha^2)t+\beta]}+\frac{1}{32}\epsilon\alpha^2\cos{3[(1+\frac{3}{8}\epsilon\alpha^2)t+3\beta]}+\ldots \eqno(13)$$
Таким образом, разложение (13) будет равномерным разложением первого порядка, поскольку секулярные члены в нём отсутсвуют, а поправка (член пропорциональный $\epsilon$) оказывается малой по сравнению с главным членом разложения.

\end{document}
