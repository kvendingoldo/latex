\documentclass[12pt]{article}
\usepackage[utf8]{inputenc}
\usepackage[T2A]{fontenc}
\usepackage[russian]{babel}

 
\usepackage{algorithm}
\usepackage{algpseudocode}

\begin{document}


\begin{algorithm}


 
\State Шаг 1: Китайская теорема об остатках \\
$$p=\prod\limits_{n = 1}^{s}p_i$$
$$ r_i=\prod\limits_{j = 1}^{s}r_{ji}$$

for $i = 1$ to s:
использовать алгоритм Евклида для нахождения \\
$p^{'}_i$ такого что $\frac{p}{p_i} \equiv 1 mod p_i$ \\
конец \\
$\overline{r}=\sum\limits_{i=1}^s\frac{p}{p_i}p_i^{'}r_j mod p$

Шаг 2: Алгоритм Евклида \\
$u_{-1}=p, u_0=\overline{r}$ \\\
$v_{-1}=0, v_0=1$ \\
$i-=1$ \\
Пока $u_i <\sqrt(p)$
$q_i=u_{i-1}/u_i$ \\
$u_{i+1}=u_{i-1}-q_iu_i$ \\
$v_{i+1}=v_{i_1} + q_iv_i$ \\
$i++$ \\
конец


Рациональное решение
$r=((-1)^iu_i/v_i)$

\end{algorithm}







	
\end{document}
