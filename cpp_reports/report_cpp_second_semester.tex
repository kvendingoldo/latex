%\documentclass{article}
\documentclass[a4paper,,titlepage]{report}
 
\usepackage[T2A]{fontenc} % Поддержка русских букв
\usepackage[utf8]{inputenc} % Кодировка utf8
\usepackage[english, russian]{babel} % Языки: русский, английский
\usepackage{pscyr} % Нормальные шрифты
\usepackage{geometry}
\usepackage{xcolor}
\usepackage{hyperref}
 
 % Цвета для гиперссылок
\definecolor{linkcolor}{HTML}{799B03} % цвет ссылок
\definecolor{urlcolor}{HTML}{799B03} % цвет гиперссылок
 \geometry{textwidth=18cm} % ширина листа
\hypersetup{pdfstartview=FitH,  linkcolor=linkcolor,urlcolor=urlcolor, colorlinks=true}

\title{Sundstrom Problem 1 -- Draft 1}
\author{Ted Sundstrom}

\renewcommand{\rmdefault}{ftm}
%ArialMT}
\renewcommand*{\thesection}{task \arabic{section}:} %переопределение
\newcommand{\gitlink}[1]{\textsf{\textbf {\textcolor{purple}{GitHub}:  }}\selectfont \href{}#1\par}
\newcommand{\comments}[1]{\textsf{\textbf {Комментарии:  }}#1\par}
\newcommand{\tests}[1]{\textsf{\textbf {Тесты:  }}#1\par}
\newcommand{\tasknumber}[1]{\textsf{\textbf {Задача: }} №#1\par}
\newcommand{\stl}[1]{\textsf{\textbf {std::}}#1\par}
%-----------------------------------------------------------------------

\begin{document}

  \begin{center}
    \large

    \vspace{0.5cm}
НИУ «Саратовский государственный университет имени Н.Г. Чернышевского»

    \vspace{0.25cm}
    
    механико-математический факультет

    \vfill
    {\LARGE Отчет по практике за 2-й семестр.\\Выполнил студент 1 курса \\Александр Шаров}

    \vfill



  \bigskip
    
    Саратов 2014
\end{center}



\begin{center}

\end{center}

\newpage
\begin{center}\section{строки I}\end{center}
 
 \begin{flushleft}
\comments{bla\newline bla}
\tests{bla} 
\gitlink{https://github.com/Harrix/Math-Harrix-Library}


\begin{table}[h]
\begin{tabular}{|l|l|l}
\cline{1-2}
\multicolumn{2}{|l|}{тесты} &  \\ \cline{1-2}
           &           &  \\ \cline{1-2}
           &           &  \\ \cline{1-2}
\end{tabular}
\end{table}


\end{flushleft}


\begin{center}\section{строки II}\end{center}
 
 \begin{flushleft}
\comments{bla\newline bla}
\tests{bla} 
\gitlink{https://github.com/Harrix/Math-Harrix-Library}
\end{flushleft}


\begin{center}\section{строки III}\end{center}
 
 \begin{flushleft}
\comments{bla\newline bla}
\tests{bla} 
\gitlink{https://github.com/Harrix/Math-Harrix-Library}
\end{flushleft}


\begin{center}\section{файлы I}\end{center}
 
 \begin{flushleft}
\comments{bla\newline bla}
\tests{bla} 
\gitlink{https://github.com/Harrix/Math-Harrix-Library}
\end{flushleft}


\begin{center}\section{файлы II}\end{center}
 
 \begin{flushleft}
\comments{bla\newline bla}
\tests{bla} 
\gitlink{https://github.com/Harrix/Math-Harrix-Library}
\end{flushleft}



\begin{center}\section{файлы III}\end{center}
 
 \begin{flushleft}
\comments{bla\newline bla}
\tests{bla} 
\gitlink{https://github.com/Harrix/Math-Harrix-Library}
\end{flushleft}


\begin{center}\section{структуры I}\end{center}
 
 \begin{flushleft}
\comments{bla\newline bla}
\tests{bla} 
\gitlink{https://github.com/Harrix/Math-Harrix-Library}
\end{flushleft}

 
\begin{center}\section{структуры II}\end{center}
 
 \begin{flushleft}
\comments{bla\newline bla}
\tests{bla} 
\gitlink{https://github.com/Harrix/Math-Harrix-Library}
\end{flushleft}


\begin{center}\section{рекурсивные функции I}\end{center}
 
 \begin{flushleft}
\comments{bla\newline bla}
\tests{bla} 
\gitlink{https://github.com/Harrix/Math-Harrix-Library}
\end{flushleft}


\begin{center}\section{рекурсивные функции I}\end{center}
 
 \begin{flushleft}
\comments{bla\newline bla}
\tests{bla} 
\gitlink{https://github.com/Harrix/Math-Harrix-Library}
\end{flushleft}

\begin{center}\section{рекурсивные функции II}\end{center}
 
 \begin{flushleft}
\comments{bla\newline bla}
\tests{bla} 
\gitlink{https://github.com/Harrix/Math-Harrix-Library}
\end{flushleft}


\begin{center}\section{рекурсивные функции III}\end{center}
 
 \begin{flushleft}
\comments{bla\newline bla}
\tests{bla} 
\gitlink{https://github.com/Harrix/Math-Harrix-Library}
\end{flushleft}


\begin{center}\section{рекурсивные функции IV}\end{center}
 
 \begin{flushleft}
\comments{bla\newline bla}
\tests{bla} 
\gitlink{https://github.com/Harrix/Math-Harrix-Library}
\end{flushleft}


\begin{center}\section{класс}\end{center}
 
 \begin{flushleft}
\comments{bla\newline bla}
\tests{bla} 
\gitlink{https://github.com/Harrix/Math-Harrix-Library}
\end{flushleft}


\begin{center}\section{стек}\end{center}
 
 \begin{flushleft}
\comments{bla\newline bla}
\tests{bla} 

\gitlink{https://github.com/Harrix/Math-Harrix-Library}
\end{flushleft}


\begin{center}\section{очередь}\end{center}
 
 \begin{flushleft}
\comments
{
\\
В тексте программы два варианта решения. 
\\Первый вариант использует алгоритм STL \textbf {std::unique}, который удаляет все повторы на промежутке [first, last). Вывод вектора осуществляется с помощью итераторов(хотя можно и без них).
\\Второй вариант - комбинация алгоритмов \textbf {std::unique, std::erase,std::sort}, а также вывод с помощью \textbf {auto(c++11)} 

}
\tests{bla} 
\gitlink{https://github.com/Harrix/Math-Harrix-Library}
\end{flushleft}


\begin{center}\section{векторы I}\end{center}
 
 \begin{flushleft}
 
 \tasknumber{18}
\comments
	{
	\\
	В тексте программы два варианта решения. 
	\\Первый вариант использует алгоритм STL \stl{unique}, который удаляет все повторы на промежутке [first, last). Вывод вектора осуществляется с помощью итераторов(хотя можно и без них).
	\\Второй вариант - комбинация алгоритмов \textbf {std::unique, std::erase,std::sort}, а также вывод с помощью \textbf {auto(c++11)} 
	}
\tests
	{
	\\ 
	\begin{table}[h]
	\begin{tabular}{|l|l|l}
	\cline{1-2}
	\multicolumn{2}{|l|}{тесты} &  \\ \cline{1-2}
		  1 2 3     &     3      &  \\ \cline{1-2}
		       &           &  \\ \cline{1-2}
	\end{tabular}
	\end{table}

	} 
\gitlink{https://raw.githubusercontent.com/kvendingoldo/report2sem/master/vec18.cpp}
\end{flushleft}


\begin{center}\section{векторы II}\end{center}
 
 \begin{flushleft}
\comments
	{
	В даннной задаче используется std::accumulate, который лежит в <numeric> и суммирует элементы в промежутке [first,last). Так же используются стандартные алгоритмы поиска итератора на максимальный и минимальный элемент( std::min element, std::max element	}
\tests{bla} 
\gitlink{https://raw.githubusercontent.com/kvendingoldo/report2sem/master/vecII.cpp}
\end{flushleft}


\begin{center}\section{списки I}\end{center}
 
 \begin{flushleft}
\comments{bla\newline bla}
\tests{bla} 
\gitlink{https://github.com/Harrix/Math-Harrix-Library}
\end{flushleft}
 

\begin{center}\section{списки II}\end{center}
 
 \begin{flushleft}
\comments{bla\newline bla}
\tests{bla} 
\gitlink{https://github.com/Harrix/Math-Harrix-Library}
\end{flushleft}
 

\end{document}