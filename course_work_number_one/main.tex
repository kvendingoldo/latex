\documentclass[bachelor, och, coursework, times]{SCWorks}

\usepackage[utf8]{inputenc}
\usepackage[T2A]{fontenc}
\usepackage[russian]{babel}

\usepackage{amsmath}
\usepackage{dsfont} % it needs for Math-special words as A
\usepackage{color} % I use it for color links
\usepackage{hyperref}


\hypersetup{
    colorlinks=true, %set true if you want colored links
    linktoc=all,     %set to all if you want both sections and subsections linked
    linkcolor=blue,  %choose some color if you want links to stand out
}


\newcommand\tab[1][1cm]{\hspace*{#1}}
\newcommand{\udsum}[3]{\sum\limits_{#1}^{#2}{#3}}
\newcommand{\dsum}[2]{\sum\limits_{#1}{#2}}
\newcommand{\tl}{\newline\tab}


%\renewcommand{\thesection}{\arabic{section}}
%\renewcommand{\thesubsection}{\arabic{section}}

\author{Sharov Alex}


\begin{document}
%\input titlepage

\chair{дифференциальных уравнений и прикладной математики}

\worktitle{Квадратичные формы и их приложения}

\course{2}

\department{Механико-математического факультета}

\group{213}

\napravlenie{01.03.02 Прикладная математика и информатика}


\studentName{Шарова Александра Вадимовича}

\chtitle{} 
\chname{}  

\satitle{} 
\saname{}  

\patitle{} 
\paname{}  

\term{} 

\duration{} 

\practStart{}  
\practFinish{} 

\year{2016}  
\MakeTitle




\setcounter{tocdepth}{1}

\tableofcontents

\intro
Данная курсовая работа освещает некоторые вопросы теории квадратичных форм.

Квадратичные формы являются очень важной вещью в приложениях и активно используются в современной физике, математике и computer science.

Открыты квадратичные формы были во время рассмотрения кривых второго порядка Лагранжем, а после расширенны многими другими математиками, в том числе Гаусс внёс огромный вклад в развитие данной теории. 

Целью работы является рассмотрения понятия квадратичных форм, связанных с этим определёний, а так же изучениям свойств. 

\section{Квадратичная форма и ее свойства}

\tab Квадратичной формой над множеством $\mathds{a}$ называют однородный полином второй степени с коэффициентами из $\mathds{a}$;
\tl
(Полином (многочлен) $F(x_1,x_2,...,x_l)$ называется однородным полиномом или формой степени (или порядка) $m$, если все его одночлены имеют степень $m$. Однородный полином первого, второго или третьего порядков называют также, соответственно, линейной, квадратичной или кубической формой.)
\tl
Если переменные квадратичной формы обозначить $x_1,...,x_n$, то общий вид квадратичной формы от этих переменных:
\\
$$f(x_1,...x_n)=\dsum{1 \le j \le k \le n}{f_{jk}x_jx_k} = \udsum{j=1}{n}{}\udsum{k=1}{n}{a_{jk}x_{j}x{k}}\eqno(1.1)$$
\\ \tab где $a_{jk} = f_{jk}$ некоторые числа, называемые коэффициентами.
\\
Не ограничивая общности, можно считать, что $a_{jk} = a_{kj}$. Квадратичная форма называется действительной или комплексной в зависимости от того, являются ли ее коэффициенты соответственно действительными или комплексными числами. Будем рассматривать действительные квадратичные формы.
\tl 
Квадратичная форма обладает следующими свойствами:
\tl
1) Для любой квадратичной формы $А$ существует единственная симметричная билинейная форма B, такая, что $A(x) = B(x, x)$. Билинейную форму B называют полярной к $A$. 
\tl
(Билинейной формой называется функция $F:L \times L \to K$, линейная по каждому из аргументов, где $L$ есть векторное пространство над полем $K$: \\
\tab $F(x+z,y)=F(x,y)+F(z,y)$ \\
\tab $F(x,y+z)=F(x,y)+F(x,z)$ \\
\tab $F(\lambda x,y)= F(x,\lambda y) = \lambda F(x,y)$, здесь $x,y,\in L$ и $\lambda \in K$)
\tl
Матрица билинейной формы в произвольном базисе совпадает с матрицей полярной ей билинейной формы в том же базисе.
\tl
(Базис — множество таких векторов в векторном пространстве, что любой вектор этого пространства может быть единственным образом представлен в виде линейной комбинации векторов из этого множества — базисных векторов.)
\tl
2) Если матрица квадратичной формы имеет полный ранг, то квадратичную форму называют невырожденной, иначе - вырожденной.
\tl
(Рангом матрицы $A$ с $m$ строк и $n$ столбцов называется максимальное число линейно независимых строк (столбцов). Несколько строк (столбцов) называются линейно независимыми, если ни одна из них не выражается линейно через другие. Ранг $rangM$ матрицы $M$ размера $m \times n$ называют полным, если $rangM = min(m,n)$.)
\tl
3) Квадратичная форма A(x,x) называется положительно (отрицательно) определённой, если для любого $x \ne 0$ $A(x,x)> 0$ $(A(x,x)<0)$. Положительно определённые и отрицательно определённые формы называются знако-определёнными. 
\tl
Квадратичная форма является положительно определенной, тогда и только тогда, когда все угловые миноры её матрицы строго положительны.
\tl
Квадратичная форма является отрицательно определенной, тогда и только тогда, когда знаки всех угловых миноров её матрицы чередуются, причем минор порядка 1 отрицателен.
\tl
(Минором k-го порядка матрицs M с  строк и  столбцов называется определитель k-го порядка, элементами которого являются элементы матрицы М, стоящих на пересечении k строк и k столбцов. 
Минор, расположенный в первых k строках и k столбцах, называется угловым минором.)
\tl
4) Квадратичная форма $A(x,x)$ называется знакопеременной, если она принимает как положительные, так и отрицательные значения.
\tl
5) Квадратичная форма $A(x,x)$ называется квазизнакоопределённой, если  $A(x,x) \ge 0 (A(x,x) \le 0)$ , но форма не является знакоопределённой.
\tl
Для приведения квадратичной формы к каноническому виду используется метод Лагранжа. Данный метод состоит в последовательном выделении в квадратичной форме полных квадратов. Возможны два случая:
\tl
1. Пусть $f_{11} \ne 0$. Выделим в $f(x_1,x_2,...x_n)$ все слагаемые, содержащие 
$f_{11}x_{1}^{2}+f_{12}x_{1}x_{2}+...+f_{1n}x_{1}x_{n}+\dsum{2 \le j \le k \le n}{f_{jk}x_{j}x_{k}}$
\tl
В последнем представлении первое слагаемое представляет собой квадрат линейной формы по переменным $x_1,x_2,...x_n$ ; все оставшиеся слагаемые не зависят от $x_1$, т.е. составляют квадратичную форму от переменных $x_2,...x_n$. Таким образом, исходная задача для формы $n$ переменных оказывается сведенной к случаю формы $(n-1)$-й переменной; последняя преобразуется по аналогичному принципу.
\tl
2. Если $f_{11}=0$, но $\exists k: f_{kk} \ne 0$ , т.е. при хотя бы одном квадрате переменной коэффициент отличен от нуля. Алгоритм модифицируется таким образом, что выделение полного квадрата начинается с переменной $x_{k}$ вместо $x_{1}$.


\section{Матрица квадратичной формы}
\tab Матрицей квадратичной формы называется матрица, составленная из ее коэффициентов. Квадратичной форме соответствует единственная симметрическая матрица (Симметрической называют квадратную матрицу, элементы которой симметричны относительно главной диагонали. Более формально, симметричной называют такую матрицу $A$ что $\forall i,j:a_{ij}=a_{ji}$.)

$$A = \begin{pmatrix}
a_{11} & a_{12} & \cdots & a_{1n} \\
a_{21} & a_{22} & \cdots & a_{2n} \\        
\vdots & \vdots & \ddots & \vdots \\
a_{n1} & a_{n2} & \cdots & a_{nn}
\end{pmatrix}\eqno(1.2)$$

где $a_{ij}=f_{ij}$
\tl
И наоборот, всякой симметрической матрице соответствует единственная квадратичная форма с точностью до обозначения переменных.
\tl
Рангом квадратичной формы называют ранг ее матрицы. Квадратичная форма $n$ переменных называется невырожденной, если ее матрица невырожденная, т. е. $r = n$, и вырожденной, если $r < n$, где $r=rang(A)$. 
\tl
Матрицей называется прямоугольная таблица из чисел, содержащая некоторое количество $m$ строк и некоторое количество $n$ столбцов. Числа $m$ и $n$ называются порядками матрицы. В случае, если $m = n$, матрица называется квадратной, а число $m= n$ – ее порядком.
\tl
В дальнейшем для записи матриц будут применяться либо сдвоенные черточки, либо круглые скобки.
\tl
Квадратичную форму $n$ переменных $x_1, x_2,...,x_n$ можно записать в матричном виде. Действительно, если $Х$ – матрица-столбец из переменных $(x_1, x_2,...,x_n)$, $x^{T}$ – матрица, полученная транспонированием матрицы $X$, т.е. матрица-строка из тех же переменных, то $f(x_1, x_2,...,x_n)= X^{T}AX$, где $A$ - это исходная матрица квадратичной формы.

\section{Преобразование квадратичной формы при линейном однородном преобразовании переменных}
\tab Рассмотрим квадратичную форму (1.1). Перейдем к новым переменным $y_1, y_2…,y_n$ по формулам (1.3):
$$x_1=b_{11}y_1+b_{12}y_2+...+b_{1n}y_n$$
$$x_2=b_{21}y_1+b_{22}y_2+...+b_{2n}y_n$$
$$...$$
$$x_n=b_{n1}y_1+b_{n2}y_2+...+b_{nn}y_n$$

или в матричном виде $X=BY$ (1.4), где  \newline
\begin{center}
$X = \begin{pmatrix}
x_{1}  \\
x_{2} \\        
\vdots \\
x_{n}
\end{pmatrix}$
$B = \begin{pmatrix}
b_{11} & b_{12} & \cdots & b_{1n} \\
b_{21} & b_{22} & \cdots & b_{2n} \\        
\vdots & \vdots & \ddots & \vdots \\
b_{n1} & b_{n2} & \cdots & b_{nn}
\end{pmatrix}$
$Y = \begin{pmatrix}
y_{1}  \\
y_{2} \\        
\vdots \\
y_{n}
\end{pmatrix}$
\end{center}
\tab
В квадратичной форме (1.1) вместо $(x_1, x_2,...,x_n)$ подставим их выражения через $(y_1, y_2,...,y_n)$ определяемые формулами (1.3), получим квадратичную форму $\phi (y_1, y_2,...,y_n)$ $n$ переменных с некоторой матрицей $С$. В этом случае говорят, что квадратичная форма $f(x_1, x_2,...,x_n)$ переводится в квадратичную форму $\phi(y_1, y_2,...,y_n)$ линейным однородным преобразованием (1.3). Линейное однородное преобразование (1.4) называется невырожденным, если $det B \ne 0$, где $det$ – определитель матрицы $B$.
\tl
Две квадратичные формы называются конгруэнтными, если существует невырожденное линейное однородное преобразование, переводящее одну форму в другую. Если $f(x_1, x_2,...,x_n)$ и $\phi(y_1, y_2,...,y_n)$ конгруэнтны, то будем писать $f(x_1, x_2,...,x_n) \sim \phi(y_1, y_2,...,y_n)$. Свойства конгруэнтности квадратичных форм:
\tl
1.  $f(x_1, x_2,...,x_n) \sim \phi(y_1, y_2,...,y_n)$.
\tl
2. Если $f(x_1, x_2,...,x_n) \sim \phi(y_1, y_2,...,y_n), \phi(y_1, y_2,...,y_n) \sim \psi(z_1,z_2,...,z_n)$, то $f(x_1, x_2,...,x_n) \sim \psi(z_1,z_2,...,z_n)$
\tl
Теорема 1. Квадратичная форма $f(x_1, x_2,...,x_n)$ с матрицей $А$ линейным однородным преобразованием $Х = ВУ$ переводится в квадратичную форму $\phi(y_1, y_2,...,y_n)$ с матрицей $C=B^{T}AB$.
\tl
Следствие 1. Определители матриц конгруэнтных невырожденных действительных квадратичных форм имеют одинаковые знаки.
\tl
Следствие 2. Конгруэнтные квадратичные формы имеют одинаковые ранги.

\section{Приведение действительной квадратичной формы к нормальному виду}
\tab Квадратичная форма $f(x_1,x_2,...,x_n)$ называется канонической, если она не содержит произведений различных переменных, т.е. $$f(x_1,x_2,...,x_n)=\udsum{i=1}{r}{a_{ii}x_i^2} (r \le n)$$
\tl
Каноническая квадратичная форма называется нормальной (или имеет нормальный вид), если $|a_n| = 1 ( i= 1, 2, . . . , r)$, т. е. отличные от нуля коэффициенты при квадратах переменных равны $+1$ или -$1$. 
\tl
Теорема 2. Любая квадратичная форма некоторым невырожденным линейным преобразованием может быть приведена к каноническому виду \newline $\phi(y_1,y_2,...,y_n)=b_{11}y_1^2+b_{22}y_2^2+...+b_{nn}y_n^2$ , где $y_1,y_2,...,y_n$ - новые переменные.
\tl
Некоторые из коэффициентов $b_{ij}$ могут оказаться равными нулю; число отличных от нуля коэффициентов в этой формуле равно рангу $r$ матрицы квадратичной формы $\phi$. 

Теорема 3. Любую действительную квадратичную форму линейным невырожденным преобразованием можно привести к нормальному виду

$\psi(z_1,z_2,...,z_n)=z_1^2+z_2^2+...+z_{k-1}^2-z_k^2-...-z_r^2$
Число входящих сюда квадратов равно рангу формы.

\section{Закон инерции квадратичных форм}
\tabЗакон инерции квадратичных форм выражает:
\tl
Теорема 4. Число положительных и число отрицательных квадратов в нормальном виде, к которому приводится данная действительная квадратичная форма невырожденным действительным линейным преобразованием, не зависит от выбора преобразования.
\tl
Число положительных квадратов в нормальной форме, к которой приводится данная действительная квадратичная форма, называют положительным индексом инерции этой формы, число отрицательных квадратов - отрицательным индексом инерции, разность между положительным и отрицательным индексами инерции - сигнатурой формы $f$. Если известен ранг формы, то задание любого из трех указанных выше чисел определяет два других.
\tl
Теорема 5. Две действительные квадратичные формы от $n$ переменных тогда и только тогда конгруэнтны, когда они имеют одинаковые ранги и одинаковые сигнатуры.
матрица квадратичный переменный


\section{Знакоопределенные квадратичные формы}
\tab Действительная квадратичная форма $f(x_1,x_2…,x_n)$ называется положительно-определенной, если она приводится к нормальному виду, состоящему из $n$ положительных квадратов: $f(x_1,x_2…,x_n) \sim \phi(y_1,y_2….y_n)$, где \newline
$$\phi(y_1,y_2, … ,y_n)=y_1^2+y_2^2+…+y_n^2$$ т.е. если ранг и положительный индекс инерции равны числу неизвестных.
\tl
Систему значений $x_1,x_2…,x_n$ назовем нулевой, если $x_1 = х_2 = ... = x_n = 0$, и ненулевой, если хотя бы одно из них отлично от нуля.
\tl
Теорема 6. Действительная квадратичная форма $f(x_1,x_2…,x_n)$ является положительно-определенной тогда и только тогда, когда она принимает положительные значения при любой ненулевой системе значений переменных $x_1,x_2…,x_n$.
\tl Пусть дана квадратичная форма $f(x_1,x_2…,x_n)$ с матрицей $A = (ay)$. Главными минорами квадратичной формы $f$ называются миноры  
$$a_{11}, 
\begin{vmatrix}
a_{11} & a_{12} \\ 
a_{21} & a_{22} \\ 
\end{vmatrix},
\cdots
,
\begin{vmatrix}
a_{11} & a_{12} & \cdots & a_{1k} \\ 
a_{21} & a_{22} & \cdots & a_{2k} \\ 
\cdots \\
a_{k1} & a_{k2} & \cdots & a_{kk} \\ 
\end{vmatrix}
,
\cdots
,
\begin{vmatrix}
a_{11} & a_{12} & \cdots & a_{1n} \\ 
a_{21} & a_{22} & \cdots & a_{2n} \\ 
\cdots \\
a_{n1} & a_{n2} & \cdots & a_{nn} \\ 
\end{vmatrix}
$$
\tab т.е. миноры порядка $1, 2, ... , n$ матрицы $А$, расположенные в левом верхнем углу; последний из них совпадает с определителем матрицы.
\tl
Теорема 7. Квадратичная форма $f(x_1,x_2…,x_n)$ с действительной матрицей является положительно-определенной тогда и только тогда, когда все ее главные миноры положительны.
\tl
Действительная квадратичная форма называется отрицательно-определенной, если она является невырожденной и приводится к нормальному виду, содержащему только отрицательные квадраты всех переменных; эту форму можно привести к виду:
$\phi(y_1,y_2….y_n)= -y_{1}^2-y_{2}^2 -…- y_{n}^2$
\tl
Теорема 8. Квадратичная форма является отрицательно-определенной тогда и только тогда, когда ее главные миноры четного порядка положительны, а нечетного - отрицательны.
\tl
Положительно-определенные и отрицательно-определенные квадратичные формы называются знакоопределенными квадратичными формами.
\tl
Вырожденные квадратичные формы, нормальный вид которых состоит из квадратов одного знака, называются полуопределенными. Неопределенными называются квадратичные формы, нормальный вид которых содержит как положительные, так и отрицательные квадраты переменных.

\section{Примеры использования}
\tab Пример 1. Найти второй дифференциал сложной функции $z(y)=f(x(y))$, где $x(y)=Sy$, в окрестности некоторого фиксированного значения векторного аргумента $y0$, если известны матрица Гессе, вычисленная при $x=x0=Sy0$, и матрица $S$
линейной замены переменных:
$$\frac{d^2f(x_0)}{dx^Tdx}
=\begin{pmatrix}
1&1\\
1&1
\end{pmatrix}\!,\quad 
S=
\begin{pmatrix}
1&2\\
1&3
\end{pmatrix}\!.$$
Решение. Второй дифференциал скалярной функции $f(x)$
$$d^2f(x)=\sum_{i=1}^{n}\sum_{j=1}^{n}\frac{\partial^2f(x)}{\partial x_i\,\partial x_j}\,dx_i\,dx_j= dx^T\,\frac{d^2f(x)}{dx^Tdx}\,dx$$
является квадратичной формой дифференциалов $dx_1,…,dx_n$
независимых переменных, причем матрица Гессе является матрицей этой квадратичной формы. При линейной замене переменных матрица Гессе функции $z(y)$ преобразуется по закону $A'=S^T\cdot A\cdot S$, т.е.
$$\frac{d^2z(y_0)}{dy^Tdy}= S^T\frac{d^2f(x_0)}{dx^Tdx}\,S= \begin{pmatrix}
1&1\\
2&3
\end{pmatrix}\!\cdot\! 
\begin{pmatrix}
1&1\\
1&1
\end{pmatrix}\!\cdot\! 
\begin{pmatrix} 
1&2\\
1&3
\end{pmatrix} = 
\begin{pmatrix} 
4&10\\
10&25
\end{pmatrix}\!.$$

Следовательно, искомый дифференциал имеет вид
$$d^2z=
\begin{pmatrix}
dy_1&dy_2
\end{pmatrix}\!\cdot\! 
\begin{pmatrix}
4&10\\
10&25 
\end{pmatrix}\!\cdot\! 
\begin{pmatrix}dy_1\\dy_2\end{pmatrix}= 4dy_1^2+20dy_1dy_2+25dy_2^2.$$

\begin{thebibliography}{00} % Список литературы
\bibitem{1}
http://mathhelpplanet.com/static.php?p=linyeinye-i-kvadratichnye-formy
\bibitem{2}
Eva Bayer-Fluckiger David Lewis, Andrew Ranicki. Quadratic Forms and Their Applications
\bibitem{3}
В.А. Ильин, Э.Г. Позняк. Линейная алгебра. М.: Наука, 1999
\bibitem{4}
Гусак А.А., Гусак Г.М., Бричикова Е.А. Справочник по высшей математике. Мн.: ТетраСистемс, 1999. - 640с. 
\bibitem{5}
Крамер Г. Математические методы статистики. – М.: Регулярная и хаотическая динамика, 2003. 
\bibitem{6}
Виноградов И.М. Элементы высшей математики. (Аналитическая геометрия. Дифференциальное исчисление. Основы теории чисел). Учебник для вузов. – М.: Высш. шк., 1999.
\end{thebibliography}

\end{document}
