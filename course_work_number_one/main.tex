\documentclass[bachelor, och, coursework, times]{SCWorks}

\usepackage[utf8]{inputenc}
\usepackage[T2A]{fontenc}
\usepackage[russian]{babel}
\usepackage{amsmath}
\usepackage{dsfont} %it needs for Math-special words as A
\usepackage{color} %I use it for color links
\usepackage{hyperref}
\usepackage{geometry}

\geometry{verbose,a4paper,tmargin=2cm,bmargin=2cm,lmargin=2.5cm,rmargin=1.5cm}

\hypersetup{
    colorlinks=true, %set true if you want colored links
    linktoc=all,     %set to all if you want both sect. and subsect. linked
    linkcolor=blue,  %choose some color if you want links to stand out
}

\newcommand\tab[1][1cm]{\hspace*{#1}}
\newcommand{\udsum}[3]{\sum\limits_{#1}^{#2}{#3}}
\newcommand{\dsum}[2]{\sum\limits_{#1}{#2}}
\newcommand{\tl}{\newline\tab}

%\renewcommand{\thesection}{\arabic{section}}
%\renewcommand{\thesubsection}{\arabic{section}}

\author{Sharov Alex}

\begin{document}

\chair{теории функций и приближений}
\worktitle{Квадратичные формы и их приложения}
\course{2}
\department{механико-математического факультета}
\group{213}
\napravlenie{01.03.02 Прикладная математика и информатика}
\studentName{Шарова Александра Вадимовича}

\satitle{к.ф-м.н., доцент} 
\saname{Смирнов А.К.}
\chtitle{д.ф-м.н., доцент} 
\chname{Сидоров С.П.}  
\patitle{} 
\paname{}
\term{} 
\practStart{}  
\practFinish{} 
\year{2016}  
\MakeTitle

\setcounter{tocdepth}{1}

\tableofcontents

\intro
В данной курсовой работе будут представлены основные вопросы и положения теории квадратичных форм.

Квадратичные формы вида $Q(x)=\udsum{j=1}{n}{}\udsum{k=1}{n}{a_{jk}x_{j}x{k}}$ очень важны и активно используются в таких областях как современная физика и математика.

В 628-м году индийский математик Брахмагупта написал \\ <<Brahmasphutasiddhanta>>, которое включало в себя уравнение вида $$x^2-ny^2=c.$$ Сейчас это уравнение называется уравнением Пелля и записывается в виде $$x^2-ny^2=1.$$ В Европе проблему решения этого уравнения изучали такие ученые как Эйлер, Лагранж, Браункер и многие другие. В частности, Лагранж в своих трудах значительно расширил теорию квадратичных форм изучая кривые второго порядка. В 1801 году Гаусс опубликовал <<Disquisitiones Arithmeticae>>, большая часть которого была посвещена  полной теории бинарных квадратичных форм над целыми числами. С тех пор эта концепция была обобщена, и связана с квадратичными числовыми полями, модулярным группами и другими областями математики.

Целью работы является рассмотрение понятия квадратичных форм, связанных с этим определёний, а так же изучениям их  свойств. 

\section{Основные понятия и определения}

\textbf{Базис} - набор $n$ векторов в $n-$мерном линейном пространстве, таких, что любой вектор пространства может быть представлен в виде некоторой их линейной комбинации, при этом ни один из базисных векторов не представим в виде линейной комбинации остальных. \cite{1}

\textbf{Матрицей} называется прямоугольная таблица из чисел, содержащая некоторое количество $m$ строк и некоторое количество $n$ столбцов. Числа $m$ и $n$ называются порядками матрицы. В случае, если $m = n$, матрица называется квадратной, а число $m= n$ – ее порядком. \cite{2}

\textbf{Определитель квадратной матрицы A} $$|A|=\det A=\begin{vmatrix}
a_{11} & a_{12} & \cdots & a_{1n} \\ 
a_{21} & a_{22} & \cdots & a_{2n} \\ 
\cdots \\
a_{n1} & a_{n2} & \cdots & a_{nn} \\ 
\end{vmatrix}
$$

\textbf{Минор} $A \begin{bmatrix} \alpha_1 & \alpha_2 \dots \alpha_k \\ \beta_1 & \beta_2 \dots \beta_k \end{bmatrix}$ матрицы $A$ - определитель такой квадратной матрицы $B$ порядка $k$ (который называется также порядком этого минора), элементы которой стоят в матрице $A$ на пересечении строк с номерами $\alpha_1, \alpha_2 \dots \alpha_k$ и столбцов с номерами $\beta_1, \beta_2 \dots \beta_k$. Если номера отмеченных строк совпадают с номерами отмеченных столбцов, то минор называется \textbf{главным}, а если отмечены первые $k$ строк и первые $k$ столбцов - \textbf{угловым} или \textbf{ведущим главным}. \cite{3}

Несколько строк (столбцов) называются линейно независимыми, если ни одна из них не выражается линейно через другие. \cite{4}

\textbf{Рангом матрицы} $A$ с $m$ строк и $n$ столбцов называется максимальное число линейно независимых строк (столбцов). Ранг $rangM$ матрицы $M$ размера $m \times n$ называют полным, если $rangM = min(m,n)$. \cite{4}

\textbf{Билинейной формой} называется функция $F:L \times L \to K$, линейная по каждому из аргументов, где $L$ есть векторное пространство над полем $K$: \\
\tab $F(x+z,y)=F(x,y)+F(z,y)$ \\
\tab $F(x,y+z)=F(x,y)+F(x,z)$ \\
\tab $F(\lambda x,y)= F(x,\lambda y) = \lambda F(x,y)$, здесь $x,y,\in L$ и $\lambda \in K$ \cite{5}


\section{Квадратичная форма и ее свойства}

\tab \textbf{Квадратичной формой над множеством $\mathds{a}$} называют однородный полином второй степени с коэффициентами из $\mathds{a}$; \cite{6}
\tl
(Полином (многочлен) $F(x_1,x_2,...,x_l)$ называется \textbf{однородным} полиномом или формой степени (или порядка) $m$, если все его одночлены имеют степень $m$. Однородный полином первого, второго или третьего порядков называют также, соответственно, линейной, квадратичной или кубической формой.)
\tl
Если переменные квадратичной формы обозначить $x_1,...,x_n$, то общий вид квадратичной формы от этих переменных будет иметь вид:
\\
$$f(x_1,...x_n)=\dsum{1 \le j \le k \le n}{f_{jk}x_jx_k} = \udsum{j=1}{n}{}\udsum{k=1}{n}{a_{jk}x_{j}x_{k}},\eqno(1.1)$$
\\ \tab где $a_{jk} = f_{jk}$ некоторые числа, называемые коэффициентами.
\\
Не ограничивая общности, можно считать, что $a_{jk} = a_{kj}$. Квадратичная форма называется действительной или комплексной в зависимости от того, являются ли ее коэффициенты соответственно действительными или комплексными числами. Будем рассматривать действительные квадратичные формы.
\tl 
Квадратичная форма обладает следующими свойствами:
\tl
1) Для любой квадратичной формы $А$ существует единственная симметричная билинейная форма B, такая, что $A(x) = B(x, x)$. Билинейную форму B называют полярной к $A$. 
\tl
Матрица билинейной формы в произвольном базисе совпадает с матрицей полярной ей билинейной формы в том же базисе.
\tl
2) Если матрица квадратичной формы имеет полный ранг, то квадратичную форму называют невырожденной, иначе - вырожденной.
\tl
3) Квадратичная форма A(x,x) называется положительно (отрицательно) определённой, если для любого $x \ne 0$ $A(x,x)> 0$ $(A(x,x)<0)$. Положительно определённые и отрицательно определённые формы называются знако-определёнными. 
\tl
Квадратичная форма является положительно определенной, тогда и только тогда, когда все угловые миноры её матрицы строго положительны.
\tl
Квадратичная форма является отрицательно определенной, тогда и только тогда, когда знаки всех угловых миноров её матрицы чередуются, причем минор порядка 1 отрицателен.
\tl
4) Квадратичная форма $A(x,x)$ называется знакопеременной, если она принимает как положительные, так и отрицательные значения.
\tl
5) Квадратичная форма $A(x,x)$ называется квазизнакоопределённой, если  $A(x,x) \ge 0 (A(x,x) \le 0)$ , но форма не является знакоопределённой.
\tl
6) Квадратичная форма $f(x_1,x_2,...,x_n)$ называется \textbf{канонической}, если она не содержит произведений различных переменных, т.е. $$f(x_1,x_2,...,x_n)=\udsum{i=1}{r}{a_{ii}x_i^2} (r \le n)$$
\tl
7) Каноническая квадратичная форма \textbf{называется нормальной} (или имеет нормальный вид), если $|a_n| = 1 ( i= 1, 2, . . . , r)$, т. е. отличные от нуля коэффициенты при квадратах переменных равны $+1$ или -$1$. 
\tl

Для приведения квадратичной формы к каноническому виду используется метод Лагранжа. Данный метод состоит в последовательном выделении в квадратичной форме полных квадратов. Возможны два случая:
\tl
1. Пусть $f_{11} \ne 0$. Выделим в $f(x_1,x_2,...x_n)$ все слагаемые, содержащие 
$f_{11}x_{1}^{2}+f_{12}x_{1}x_{2}+...+f_{1n}x_{1}x_{n}+\dsum{2 \le j \le k \le n}{f_{jk}x_{j}x_{k}}$
\tl
В последнем представлении первое слагаемое представляет собой квадрат линейной формы по переменным $x_1,x_2,...x_n$ ; все оставшиеся слагаемые не зависят от $x_1$, т.е. составляют квадратичную форму от переменных $x_2,...x_n$. Таким образом, исходная задача для формы $n$ переменных оказывается сведенной к случаю формы $(n-1)$-й переменной; последняя преобразуется по аналогичному принципу.
\tl
2. Если $f_{11}=0$, но $\exists k: f_{kk} \ne 0$ , т.е. при хотя бы одном квадрате переменной коэффициент отличен от нуля. Алгоритм модифицируется таким образом, что выделение полного квадрата начинается с переменной $x_{k}$ вместо $x_{1}$.


\section{Матрица квадратичной формы}
\tab \textbf{Матрицей квадратичной формы} называется матрица, составленная из ее коэффициентов. Квадратичной форме соответствует единственная симметрическая матрица (Симметрической называют квадратную матрицу, элементы которой симметричны относительно главной диагонали. Более формально, симметричной называют такую матрицу $A$ что $\forall i,j:a_{ij}=a_{ji}$.)

$$A = \begin{pmatrix}
a_{11} & a_{12} & \cdots & a_{1n} \\
a_{21} & a_{22} & \cdots & a_{2n} \\        
\vdots & \vdots & \ddots & \vdots \\
a_{n1} & a_{n2} & \cdots & a_{nn}
\end{pmatrix}\eqno(1.2)$$

где $a_{ij}=f_{ij}$ \cite{gusak}
\tl
И наоборот, всякой симметрической матрице соответствует единственная квадратичная форма с точностью до обозначения переменных.
\tl
Рангом квадратичной формы называют ранг ее матрицы. Квадратичная форма $n$ переменных называется невырожденной, если ее матрица невырожденная, т. е. $r = n$, и вырожденной, если $r < n$, где $r=rang(A)$. 
\tl
В дальнейшем для записи матриц будут применяться либо сдвоенные черточки, либо круглые скобки.
\tl
Квадратичную форму $n$ переменных $x_1, x_2,...,x_n$ можно записать в матричном виде. Действительно, если $Х$ – матрица-столбец из переменных $(x_1, x_2,...,x_n)$, $x^{T}$ – матрица, полученная транспонированием матрицы $X$, т.е. матрица-строка из тех же переменных, то $f(x_1, x_2,...,x_n)= X^{T}AX$, где $A$ - это исходная матрица квадратичной формы.

\section{Преобразование квадратичной формы при линейном однородном преобразовании переменных}
\tab Рассмотрим квадратичную форму (1.1). Перейдем к новым переменным $y_1, y_2…,y_n$ по нижеследующим формулам (1.3):
$$x_1=b_{11}y_1+b_{12}y_2+...+b_{1n}y_n$$
$$x_2=b_{21}y_1+b_{22}y_2+...+b_{2n}y_n$$
$$...$$
$$x_n=b_{n1}y_1+b_{n2}y_2+...+b_{nn}y_n$$

или в матричном виде $X=BY$ (1.4), где  \newline
\begin{center}
$X = \begin{pmatrix}
x_{1}  \\
x_{2} \\        
\vdots \\
x_{n}
\end{pmatrix}$
$B = \begin{pmatrix}
b_{11} & b_{12} & \cdots & b_{1n} \\
b_{21} & b_{22} & \cdots & b_{2n} \\        
\vdots & \vdots & \ddots & \vdots \\
b_{n1} & b_{n2} & \cdots & b_{nn}
\end{pmatrix}$
$Y = \begin{pmatrix}
y_{1}  \\
y_{2} \\        
\vdots \\
y_{n}
\end{pmatrix}$
\end{center}
\tab
В квадратичной форме (1.1) вместо $(x_1, x_2,...,x_n)$ подставим их выражения через $(y_1, y_2,...,y_n)$ определяемые формулами (1.3), получим квадратичную форму $\phi (y_1, y_2,...,y_n)$ $n$ переменных с некоторой матрицей $С$. В этом случае говорят, что квадратичная форма $f(x_1, x_2,...,x_n)$ переводится в квадратичную форму $\phi(y_1, y_2,...,y_n)$ линейным однородным преобразованием (1.3). Линейное однородное преобразование (1.4) называется невырожденным, если $det B \ne 0$, где $det$ – определитель матрицы $B$.
\tl
Две квадратичные формы называются \textbf{конгруэнтными}, если существует невырожденное линейное однородное преобразование, переводящее одну форму в другую. Если $f(x_1, x_2,...,x_n)$ и $\phi(y_1, y_2,...,y_n)$ конгруэнтны, то будем писать $f(x_1, x_2,...,x_n) \sim \phi(y_1, y_2,...,y_n)$. Свойства конгруэнтности квадратичных форм:
\tl
1.  $f(x_1, x_2,...,x_n) \sim \phi(y_1, y_2,...,y_n)$.
\tl
2. Если $f(x_1, x_2,...,x_n) \sim \phi(y_1, y_2,...,y_n), \phi(y_1, y_2,...,y_n) \sim \psi(z_1,z_2,...,z_n)$, то $f(x_1, x_2,...,x_n) \sim \psi(z_1,z_2,...,z_n)$

Все вышеперечисленное в конечном итоге приводит нас к теореме:

\textbf{Теорема 1.1.} Квадратичная форма $f(x_1, x_2,...,x_n)$ с матрицей $А$ линейным однородным преобразованием $Х = ВУ$ переводится в квадратичную форму $\phi(y_1, y_2,...,y_n)$ с матрицей $C=B^{T}AB$.
\tl
\textbf{Следствие 1.} Определители матриц конгруэнтных невырожденных действительных квадратичных форм имеют одинаковые знаки.
\tl
\textbf{Следствие 2.} Конгруэнтные квадратичные формы имеют одинаковые ранги.

\section{Ортогональное преобразование квадратичной формы к каноническому виду}

Для приведения квадратичной формы к каноническому виду существуюет достаточно простой метод, называемый методом Лагранжа.

\textbf{Теорема 2.1 (Лагранжа).} Всякая квадратичная форма при помощи невырожденного линейного преобразования переменных может быть приведена к каноническому виду. 


Этот метод, однако, во многих задачах не дает нужного результата. Например, в задачах аналитической геометрии часто требуется привести общее уравнение кривой или поверхности второго порядка к каноническому виду, причем такое приведение требуется осуществить с помощью весьма специального преобразования переменных. Метод Лагранжа не всегда обеспечивает это условие в отличии от способа основанном на отыскании собственных значений матрицы квадратичной формы. 


\textbf{Теорема 2.2.} Всякая квадратичная форма с матрицей $А$ может быть приведена к каноническому виду 
$$ \lambda_1y_1^2+\lambda_2y_2^2+\dots + \lambda_ny_n^2$$
при помощи преобразования переменных с ортогональной матрицей. При этом коэффициенты $ \lambda_k $ канонического вида являются корнями характеристического многочлена матрицы $A$ каждый из которых взят столько раз, какова его кратность. 

Можно указать практическую схему для отыскания ортогонального преобразования переменных, в результате которого квадратичная форма принимает канонический вид, или, что то же, ее матрица заменяется на диагональную. 

1-й шаг. Для данной квадратичной формы строим ее симметрическую матрицу $A$.

2-й шаг. Составляем характеристический многочлен $\Delta(\lambda) = |A - \lambda\cdot E|$ и находим его корни. Обозначим корни характеристического многочлена через $\lambda_1, \lambda_2, \dots, \lambda_n.$

3-й шаг. Зная корни характеристического многочлена $\Delta(\lambda)$ можно написать канонический вид данной квадратичной формы: 
$$ 	\tilde{f}=\lambda_1y_1^2+\lambda_2y_2^2+\dots, + \lambda_ny_n^2 $$

4-й шаг. Для каждого корня $\lambda_i$ кратности $m_i$ составляем однородную систему линейных уравнений: 

$$
\begin{cases} 
(a_{11}-\lambda_i)\zeta_1 + a_{12}\zeta_2 + \dots + a_{1n}\zeta_n = 0 \\ 
a_{21}\zeta_1 + (a_{12}-\lambda_i)\zeta_2 + \dots + a_{1n}\zeta_n = 0 \\
\dots \\
a_{n1}\zeta_1 + a_{12}\zeta_2 + \dots + (a_{1n}-\lambda_i)\zeta_n = 0
\end{cases} \eqno(5.1)
$$

где $a_{ij}$ - элементы матрицы $A$.

5-й шаг. Для каждого $\lambda_i$ кратности $m_i$ находим какую-нибудь одну ортонормированную систему из $m_i$ векторов, являющихся решениями системы (5.1). Индекс $i$ меняется от $1$ до $k$ где $k$ есть число различных корней характеристического многочлена $\Delta(\lambda)$. Т.о. получим $n$ попарно ортогональных нормированных векторов: 
$$e_1^{'}=(q_{11}, q_{21}, \dots, q_{n1}),$$
$$e_2^{'}=(q_{12}, q_{22}, \dots, q_{n2}),$$
$$\dots,$$
$$e_n^{'}=(q_{1n}, q_{2n}, \dots, q_{nn})$$

(Порядок следования векторов $e_1^{'}, e_2^{'}, \dots, e_n^{'}$ соответствует порядку $\lambda_i$ в каноническом виде.) 

6-й шаг. Составляем матрицу $Q$ столбцами которой являются координаты векторов $e_1^{'}, e_2^{'}, \dots, e_n^{'}$:

$$Q = \begin{pmatrix}
q_{11} & q_{12} & \cdots & q_{1n} \\
q_{21} & q_{22} & \cdots & q_{2n} \\        
\vdots & \vdots & \ddots & \vdots \\
q_{n1} & q_{n2} & \cdots & q_{nn}
\end{pmatrix}$$

7-й шаг. Записываем искомое ортогональное преобразование \mbox{переменных}: 

$$
\begin{cases} 
x_1 = q_{11}y_1 + q_{12}y_2 + \dots + q_{1n}y_n \\ 
x_2 = q_{21}y_1 + q_{22}y_2 + \dots + q_{2n}y_n \\
\dots \\
x_n = q_{n1}y_1 + q_{n2}y_2 + \dots + q_{nn}y_n
\end{cases}
$$

т.е.

$$\begin{pmatrix}
x_{1}  \\
x_{2} \\        
\vdots \\
x_{n}
\end{pmatrix} = Q^{'}\cdot\begin{pmatrix}
y_{1}  \\
y_{2} \\        
\vdots \\
y_{n}
\end{pmatrix}$$

8-й шаг. Если требуется выразить новые переменные $y_1, y_2, \dots, y_n$ через старые $x_1, x_2, \dots, x_n$ то, учитывая, что $Q^{-1} = Q^{'}$ получаем: 

$$Q^{'}\cdot\begin{pmatrix}
x_{1}  \\
x_{2} \\        
\vdots \\
x_{n}
\end{pmatrix} = \begin{pmatrix}
y_{1}  \\
y_{2} \\        
\vdots \\
y_{n}
\end{pmatrix}$$

Замечание. В случае правильности полученного результата должно быть $B=Q^{'}AQ$, где $B$ — диагональная матрица, отвечающая форме $\tilde{f}$ Отметим еще, что в связи с неоднозначностью отыскания фундаментальной системы решений однородной линейной системы (5-й шаг) ортогональное преобразование переменных будет находиться также неоднозначно.

\section{Закон инерции квадратичных форм}
\tabЗакон инерции квадратичных форм выражает:
\tl
\textbf{Теорема 6.1.} Число положительных и число отрицательных квадратов в нормальном виде, к которому приводится данная действительная квадратичная форма невырожденным действительным линейным преобразованием, не зависит от выбора преобразования.
\tl
Число положительных квадратов в нормальной форме, к которой приводится данная действительная квадратичная форма, называют \textbf{положительным индексом инерции} этой формы, число отрицательных квадратов - \textbf{отрицательным индексом инерции}, разность между положительным и отрицательным индексами инерции - \textbf{сигнатурой формы} $f$. Если известен ранг формы, то задание любого из трех указанных выше чисел определяет два других.
\tl
\textbf{Теорема 6.2.} Две действительные квадратичные формы от $n$ переменных тогда и только тогда конгруэнтны, когда они имеют одинаковые ранги и одинаковые сигнатуры.
матрица квадратичный переменный


\section{Знакоопределенные квадратичные формы}
\tab Действительная квадратичная форма $f(x_1,x_2…,x_n)$ называется\\ \textbf{положительно-определенной}, если она приводится к нормальному виду, состоящему из $n$ положительных квадратов: $f(x_1,x_2…,x_n) \sim \phi(y_1,y_2….y_n)$, где \newline
$$\phi(y_1,y_2, … ,y_n)=y_1^2+y_2^2+…+y_n^2$$ т.е. если ранг и положительный индекс инерции равны числу неизвестных.
\tl
Систему значений $x_1,x_2…,x_n$ назовем нулевой, если $x_1 = х_2 = ... = x_n = 0$, и ненулевой, если хотя бы одно из них отлично от нуля.
\tl
\textbf{Теорема 7.1.} Действительная квадратичная форма $f(x_1,x_2…,x_n)$ является положительно-определенной тогда и только тогда, когда она принимает положительные значения при любой ненулевой системе значений переменных $x_1,x_2…,x_n$. \cite{berezina}
\tl Пусть дана квадратичная форма $f(x_1,x_2…,x_n)$ с матрицей $A = (ay)$. Главными минорами квадратичной формы $f$ называются миноры  
$$a_{11}, 
\begin{vmatrix}
a_{11} & a_{12} \\ 
a_{21} & a_{22} \\ 
\end{vmatrix},
\cdots
,
\begin{vmatrix}
a_{11} & a_{12} & \cdots & a_{1k} \\ 
a_{21} & a_{22} & \cdots & a_{2k} \\ 
\cdots \\
a_{k1} & a_{k2} & \cdots & a_{kk} \\ 
\end{vmatrix}
,
\cdots
,
\begin{vmatrix}
a_{11} & a_{12} & \cdots & a_{1n} \\ 
a_{21} & a_{22} & \cdots & a_{2n} \\ 
\cdots \\
a_{n1} & a_{n2} & \cdots & a_{nn} \\ 
\end{vmatrix}
$$
\tab т.е. миноры порядка $1, 2, ... , n$ матрицы $А$, расположенные в левом верхнем углу; последний из них совпадает с определителем матрицы.
\tl
\textbf{Теорема 7.2.} Квадратичная форма $f(x_1,x_2…,x_n)$ с действительной матрицей является положительно-определенной тогда и только тогда, когда все ее главные миноры положительны.
\tl
Действительная квадратичная форма называется отрицательно-определенной, если она является невырожденной и приводится к нормальному виду, содержащему только отрицательные квадраты всех переменных; эту форму можно привести к виду:
$\phi(y_1,y_2….y_n)= -y_{1}^2-y_{2}^2 -…- y_{n}^2$
\tl
\textbf{Теорема 7.3.} Квадратичная форма является отрицательно-определенной тогда и только тогда, когда ее главные миноры четного порядка положительны, а нечетного - отрицательны. \cite{berezina}
\tl
Положительно-определенные и отрицательно-определенные квадратичные формы называются \textbf{знакоопределенными квадратичными формами}.
\tl
Вырожденные квадратичные формы, нормальный вид которых состоит из квадратов одного знака, называются \textbf{полуопределенными}. \textbf{Неопределенными} называются квадратичные формы, нормальный вид которых содержит как положительные, так и отрицательные квадраты переменных. \cite{berezina}

\section{Примеры использования}
\tab \textbf{Пример 1.} Найти второй дифференциал сложной функции $z(y)=f(x(y))$, где $x(y)=Sy$, в окрестности некоторого фиксированного значения векторного аргумента $y0$, если известны матрица Гессе, вычисленная при $x=x0=Sy0$, и матрица $S$
линейной замены переменных:
$$\frac{d^2f(x_0)}{dx^Tdx}
=\begin{pmatrix}
1&1\\
1&1
\end{pmatrix}\!,\quad 
S=
\begin{pmatrix}
1&2\\
1&3
\end{pmatrix}\!.$$
Решение. Второй дифференциал скалярной функции $f(x)$
$$d^2f(x)=\sum_{i=1}^{n}\sum_{j=1}^{n}\frac{\partial^2f(x)}{\partial x_i\,\partial x_j}\,dx_i\,dx_j= dx^T\,\frac{d^2f(x)}{dx^Tdx}\,dx$$
является квадратичной формой дифференциалов $dx_1,…,dx_n$
независимых переменных, причем матрица Гессе является матрицей этой квадратичной формы. При линейной замене переменных матрица Гессе функции $z(y)$ преобразуется по закону $A'=S^T\cdot A\cdot S$, т.е.
$$\frac{d^2z(y_0)}{dy^Tdy}= S^T\frac{d^2f(x_0)}{dx^Tdx}\,S= \begin{pmatrix}
1&1\\
2&3
\end{pmatrix}\!\cdot\! 
\begin{pmatrix}
1&1\\
1&1
\end{pmatrix}\!\cdot\! 
\begin{pmatrix} 
1&2\\
1&3
\end{pmatrix} = 
\begin{pmatrix} 
4&10\\
10&25
\end{pmatrix}\!.$$

Следовательно, искомый дифференциал имеет ви
$$d^2z=
\begin{pmatrix}
dy_1&dy_2
\end{pmatrix}\!\cdot\! 
\begin{pmatrix}
4&10\\
10&25 
\end{pmatrix}\!\cdot\! 
\begin{pmatrix}dy_1\\dy_2\end{pmatrix}= 4dy_1^2+20dy_1dy_2+25dy_2^2.$$

\conclusion
В ходе написания данного курсового проекта были подробно рассмотрены основные вопросы и положения теории квадратичный форм.
\tl
При написании курсовой работы была изучена специальная литература, включающая научные статьи по теории квадратичных форм, учебники по линейной алгебре, рассмотрено практическое применение квадратичных форм в приложениях.

\begin{thebibliography}{00} % Список литературы
\bibitem{1}
Mathmetod [Электронный ресурс]:\\
URL: \href{http://mathmetod.wikispaces.com/Базис}{http://mathmetod.wikispaces.com/Базис} \\ (дата обращения 03.05.2016).
\bibitem{2}
100formul [Электронный ресурс]:\\
URL: \href{ttp://100formul.ru/matrix1}{ttp://100formul.ru/matrix1} 
\\ (дата обращения 04.05.2016).
\bibitem{3}
Википедия [Электронный ресурс]:\\
URL: \href{https://goo.gl/d9FnPw}{https://goo.gl/d9FnPw}
\\ (дата обращения 02.05.2016).
\bibitem{4}
Википедия [Электронный ресурс]:\\
URL: \href{https://goo.gl/EzDACj}{https://goo.gl/EzDACj}
\\ (дата обращения 04.05.2016).
\bibitem{5}
Википедия [Электронный ресурс]:\\
URL: \href{https://goo.gl/pASYNq}{https://goo.gl/pASYNq}
\\ (дата обращения 05.05.2016).
\bibitem{6}
http://pmpu.ru [Электронный ресурс]:\\
URL: \href{http://pmpu.ru/vf4/2form}{http://pmpu.ru/vf4/2form}
\\ (дата обращения 04.05.2016).
\bibitem{s}
Виноградов И.М. Элементы высшей математики. (Аналитическая геометрия. Дифференциальное исчисление. Основы теории чисел). Учебник для вузов. – М.: Высш. шк., 1999. - 511c.
\bibitem{gusak}
Гусак А.А., Гусак Г.М., Бричикова Е.А. Справочник по высшей математике. - Мн.: ТетраСистемс, 1999. - 640с. 
\bibitem{berezena}
Березина Н. Линейная алгебра. Конспект лекций. - М.: Эксмо, 2007. - 128c.
\bibitem{2.1}
Громов А. П. Учебное пособие по линейной алгебре. - М: Изд-во "Просвещение", 1971. - 128c.
\bibitem{asa}
Крамер Г. Математические методы статистики. – М.: Регулярная и хаотическая динамика, 2003. - 648c.
\bibitem{asa}
Ильин В.А., Позняк Э.Г. Линейная алгебра. - М.: Наука, 1999. - 280c.
\bibitem{kk}
Квадратичные формы [Электронный ресурс]:\\
URL: \href{http://mathhelpplanet.com/static.php?p=linyeinye-i-kvadratichnye-formy}{http://mathhelpplanet.com/static.php?p=linyeinye-i-kvadratichnye-formy} (дата обращения 12.04.2016).
\bibitem{7}
Квадратичные формы [Электронный ресурс]:\\
URL: \href{https://en.wikipedia.org/wiki/Quadratic\underline{ }form}{https://en.wikipedia.org/wiki/Quadratic\underline{ }form} \\ (дата обращения 01.05.2016).

\end{thebibliography}

\end{document}