\documentclass[10pt,pdf,hyperref={unicode}]{beamer}

\usepackage{lmodern}
\usepackage[T2A]{fontenc}
\usepackage[utf8]{inputenc}

% отключить клавиши навигации
\setbeamertemplate{navigation symbols}{}

% тема оформления
\usetheme{CambridgeUS}

% цветовая схема
\usecolortheme{dolphin}

\title[Квадратичные формы и их прил.]{
	Квадратичные формы и их приложения
}

\author[Шаров А.В.]{
  Шаров Александр Вадимович
}
\institute[КОД] {
«Прикладная информатика и математика»
}

\date{} 
% \logo{\includegraphics[height=5mm]{images/logo.png}\vspace{-7pt}}

\begin{document}

\begin{frame}
\titlepage
\end{frame} 

\section<presentation>*{Содержание}
\begin{frame}
\frametitle{Содержание}
\tableofcontents
\end{frame} 

\section{Постановка задачи}
\begin{frame}

Целью работы является рассмотрение понятия квадратичных форм, связанных с этим определёний, а так же изучениям их  свойств. 

\end{frame} 

\section{Историческая справка}
\begin{frame}

В 628-м году индийский математик Брахмагупта написал \\ <<Brahmasphutasiddhanta>>, которое включало в себя уравнение вида $$x^2-ny^2=c.$$ Сейчас это уравнение называется уравнением Пелля и записывается в виде $$x^2-ny^2=1.$$ В Европе проблему решения этого уравнения изучали такие ученые как Эйлер, Лагранж, Браункер и многие другие. В частности, Лагранж в своих трудах значительно расширил теорию квадратичных форм изучая кривые второго порядка. В 1801 году Гаусс опубликовал <<Disquisitiones Arithmeticae>>, большая часть которого была посвещена  полной теории бинарных квадратичных форм над целыми числами. С тех пор эта концепция была обобщена, и связана с квадратичными числовыми полями, модулярным группами и другими областями математики.

\end{frame} 

\section{Основные результаты}
\begin{frame}

\begin{block}{Теорема 1}
Квадратичная форма $f(x_1, x_2,...,x_n)$ с матрицей $A$ линейным однородным преобразованием $X = BY$ переводится в квадратичную форму $\phi(y_1, y_2,...,y_n)$ с матрицей $C=B^{T}AB$.
\end{block}

\begin{block}{Следствие 1.} 
Определители матриц конгруэнтных невырожденных действительных квадратичных форм имеют одинаковые знаки.
\end{block}

\begin{block}{Следствие 2.} 
Конгруэнтные квадратичные формы имеют одинаковые ранги.
\end{block}

\end{frame} 

\begin{frame}

\begin{block}{Теорема 2 (Лагранжа).} 
Всякая квадратичная форма при помощи невырожденного линейного преобразования переменных может быть приведена к каноническому виду. 
\end{block}


\begin{block}{Теорема 3} 
Всякая квадратичная форма с матрицей $A$ может быть приведена к каноническому виду 
$$ \lambda_1y_1^2+\lambda_2y_2^2+\dots + \lambda_ny_n^2$$
при помощи преобразования переменных с ортогональной матрицей. При этом коэффициенты $ \lambda_k $ канонического вида являются корнями характеристического многочлена матрицы $A$ каждый из которых взят столько раз, какова его кратность. 
\end{block}

\end{frame} 

\begin{frame}

\begin{block}{Теорема 4} 
Число положительных и число отрицательных квадратов в нормальном виде, к которому приводится данная действительная квадратичная форма невырожденным действительным линейным преобразованием, не зависит от выбора преобразования.
\end{block}

\begin{block}{Теорема 5} 
Две действительные квадратичные формы от $n$ переменных тогда и только тогда конгруэнтны, когда они имеют одинаковые ранги и одинаковые сигнатуры.
\end{block}

\end{frame} 

\begin{frame}

\begin{block}{Теорема 6} 
Действительная квадратичная форма $f(x_1,x_2…,x_n)$ является положительно-определенной тогда и только тогда, когда она принимает положительные значения при любой ненулевой системе значений переменных $x_1,x_2…,x_n$. 
\end{block}

\begin{block}{Теорема 7} 
Квадратичная форма $f(x_1,x_2…,x_n)$ с действительной матрицей является положительно-определенной тогда и только тогда, когда все ее главные миноры положительны.
\end{block}

\begin{block}{Теорема 8} 
Квадратичная форма является отрицательно-определенной тогда и только тогда, когда ее главные миноры четного порядка положительны, а нечетного - отрицательны.
\end{block}

\end{frame} 

\section{Заключение}
\begin{frame}
-
\end{frame} 

\end{document}
