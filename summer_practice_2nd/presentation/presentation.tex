\documentclass[10pt,pdf,hyperref={unicode}]{beamer}

\usepackage{lmodern}
\usepackage{amsmath}
\usepackage{dsfont}
\usepackage[T2A]{fontenc}
\usepackage[utf8]{inputenc}

% отключить клавиши навигации
\setbeamertemplate{navigation symbols}{}

% тема оформления
\usetheme{CambridgeUS}

% цветовая схема
\usecolortheme{dolphin}

\title[Квадратичные формы и их прил.]{
	Квадратичные формы и их приложения
}

\author[Шаров А.В.]{
  Шаров Александр Вадимович
}
\institute[01.03.02] {
«01.03.02 Прикладная математика и информатика»
\\
Научный руководитель А.К. Смирнов
}

\date{7 Cентября 2016 г.} 
% \logo{\includegraphics[height=5mm]{images/logo.png}\vspace{-7pt}}

\begin{document}

\begin{frame}
\titlepage
\end{frame} 

\section<presentation>*{Содержание}
\begin{frame}
\frametitle{Содержание}
\tableofcontents
\end{frame} 

\section{ВВЕДЕНИЕ}
\begin{frame}

\begin{block}{Историческая справка}

В 628-м году индийский математик Брахмагупта написал \\ <<Brahmasphutasiddhanta>>, которое включало в себя уравнение вида $$x^2-ny^2=c.$$ Сейчас это уравнение называется уравнением Пелля и записывается в виде $$x^2-ny^2=1.$$ В Европе проблему решения этого уравнения изучали такие ученые как Эйлер, Лагранж, Браункер и многие другие. В частности, Лагранж в своих трудах значительно расширил теорию квадратичных форм изучая кривые второго порядка. В 1801 году Гаусс опубликовал <<Disquisitiones Arithmeticae>>, большая часть которого была посвещена  полной теории бинарных квадратичных форм над целыми числами. С тех пор эта концепция была обобщена, и связана с квадратичными числовыми полями, модулярным группами и другими областями математики.

\end{block}

%\begin{block}{В данной курсовой работе}

%моей задачей было изучение квадратичных форм, связанн	ых с этим определёний. Нужно было изучить основные результаты и понять, в каких именно областях можно применять квадратичные формы. 

%\end{block}

\end{frame} 

\section{Общий вид квадратичной формы}
\begin{frame}

\begin{block}{Определение}
Квадратичной формой над множеством $\mathds{a}$ называют однородный полином второй степени с коэффициентами из $\mathds{a}$;
\\
Если переменные квадратичной формы обозначить $x_1,...,x_n$, то общий вид квадратичной формы от этих переменных будет иметь вид:
\\
$$f(x_1,...x_n)=\sum_{1 \le j \le k \le n}{f_{jk}x_jx_k} = \sum_{j=1}^{n}{}\sum_{k=1}^{n}{a_{jk}x_{j}x_{k}} \;,$$
\\ где $a_{jk} = f_{jk}$ некоторые числа, называемые коэффициентами.x
\end{block}

\end{frame}


\section{Основные свойства квадратичных форм}

\begin{frame}{Основные свойства квадратичных форм}

\begin{block}

\begin{itemize}
\item{Для любой квадратичной формы $А$ существует единственная симметричная билинейная форма B, такая, что $A(x) = B(x, x)$. Билинейную форму B называют полярной к $A$.}
\item{Если матрица квадратичной формы имеет полный ранг, то квадратичную форму называют невырожденной, иначе - вырожденной.}
\item{Квадратичная форма A(x,x) называется положительно (отрицательно) определённой, если для любого $x \ne 0$ $A(x,x)> 0$ $(A(x,x)<0)$. Положительно определённые и отрицательно определённые формы называются знако-определёнными.}

\end{itemize}

\end{block}

\end{frame}

\begin{frame}{Основные свойства квадратичных форм}

\begin{block}

\begin{itemize}

\item{Квадратичная форма $A(x,x)$ называется знакопеременной, если она принимает как положительные, так и отрицательные значения.}
\item{Квадратичная форма $f(x_1,x_2,...,x_n)$ называется канонической, если она не содержит произведений различных переменных, т.е. $$f(x_1,x_2,...,x_n)=\sum_{i=1}^{r}{a_{ii}x_i^2} (r \le n)$$}
\item{Каноническая квадратичная форма называется нормальной (или имеет нормальный вид), если $|a_n| = 1 ( i= 1, 2, . . . , r)$, т. е. отличные от нуля коэффициенты при квадратах переменных равны $+1$ или -$1$. }

\end{itemize}

\end{block}

\end{frame}

\section{Основные результаты}
\begin{frame}{Преобразование квадратичной формы при линейном однородном
преобразовании переменных}

\begin{block}{Теорема 1}
Квадратичная форма $f(x_1, x_2,...,x_n)$ с матрицей $A$ линейным однородным преобразованием $X = BY$ переводится в квадратичную форму $\phi(y_1, y_2,...,y_n)$ с матрицей $C=B^{T}AB$.
\end{block}

\begin{block}{Следствие 1.} 
Определители матриц конгруэнтных невырожденных действительных квадратичных форм имеют одинаковые знаки.
\end{block}

\begin{block}{Следствие 2.} 
Конгруэнтные квадратичные формы имеют одинаковые ранги.
\end{block}

\end{frame} 

\begin{frame}{Ортогональное преобразование квадратичной формы к
каноническому виду}

\begin{block}{Теорема 2 (Лагранжа).} 
Всякая квадратичная форма при помощи невырожденного линейного преобразования переменных может быть приведена к каноническому виду. 
\end{block}


\begin{block}{Теорема 3} 
Всякая квадратичная форма с матрицей $A$ может быть приведена к каноническому виду 
$$ \lambda_1y_1^2+\lambda_2y_2^2+\dots + \lambda_ny_n^2$$
при помощи преобразования переменных с ортогональной матрицей. При этом коэффициенты $ \lambda_k $ канонического вида являются корнями характеристического многочлена матрицы $A$ каждый из которых взят столько раз, какова его кратность. 
\end{block}

\end{frame} 

\begin{frame}{Закон инерции квадратичных форм}

\begin{block}{Теорема 4} 
Число положительных и число отрицательных квадратов в нормальном виде, к которому приводится данная действительная квадратичная форма невырожденным действительным линейным преобразованием, не зависит от выбора преобразования.
\end{block}

\begin{block}{Теорема 5} 
Две действительные квадратичные формы от $n$ переменных тогда и только тогда конгруэнтны, когда они имеют одинаковые ранги и одинаковые сигнатуры.
\end{block}

\end{frame} 

\begin{frame}{Знакоопределенные квадратичные формы}

\begin{block}{Теорема 6} 
Действительная квадратичная форма $f(x_1,x_2…,x_n)$ является положительно-определенной тогда и только тогда, когда она принимает положительные значения при любой ненулевой системе значений переменных $x_1,x_2…,x_n$. 
\end{block}

\begin{block}{Теорема 7} 
Квадратичная форма $f(x_1,x_2…,x_n)$ с действительной матрицей является положительно-определенной тогда и только тогда, когда все ее главные миноры положительны.
\end{block}

\begin{block}{Теорема 8} 
Квадратичная форма является отрицательно-определенной тогда и только тогда, когда ее главные миноры четного порядка положительны, а нечетного - отрицательны.
\end{block}

\end{frame} 

\section{ЗАКЛЮЧЕНИЕ}
\begin{frame}{Заключение}
\begin{block}

\begin{itemize}

\item{Подробно рассмотрены основные вопросы и положения теории квадратичный форм.}
\item{Изучена специальная литература, включающая научные статьи по теории квадратичных форм, учебники по линейной алгебре.}
\item{Рассмотрено практическое применение квадратичных форм в приложениях.}


\end{itemize}

\end{block}
\end{frame} 

\end{document}
