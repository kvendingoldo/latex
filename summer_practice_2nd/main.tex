\documentclass[article, bachelor, och, pract]{SCWorks}


\usepackage[utf8]{inputenc}
%\usepackage[T2A]{fontenc}
\usepackage{amsthm,amsmath,amssymb,amsfonts}
\usepackage[russian]{babel}
\usepackage{indentfirst}
%%%%%%%%%%%%%%%%%%%%%
\usepackage{graphicx}
\graphicspath{{pictures/}}
\DeclareGraphicsExtensions{.png,.jpg}


%\usepackage[colorlinks=true]{hyperref}

\begin{document}

\chair{дифференциальных уравнений и прикладной математики}
\worktitle{}
\course{2}

\group{213}




\chtitle{д.ф-м.н., доцент} 
\chname{}  
%\patitle{} 
%\paname{}
%\term{} 



\napravlenie{01.03.02 "Прикладная математика и информатика"}

\studentName{Шарова Александра Вадимовича}

\practStart{1 Июля 2016}  
\practFinish{14 Июля 2016} 

\satitle{к.ф.-м.н., доцент}
\saname{В. А. Халова}

\year{2016}  
\MakeTitle

\setcounter{tocdepth}{1}

\tableofcontents
\intro


Цель данной практики - изучить основы работы в системе компьютерной математике Maxima. 


\section{Простые вычисления}
Упростить выражения: № 1.013, № 1.028, № 1.043 [Из сборника задач по математике для поступающих во втузы, под редакцией Сканави М. И.]. Для упрощения выражения воспользуемся командой \textbf{ratsimp(Функция)}.
\\

№ 1.013\\
Упростить выражение:

$(x^2+2x-\frac{11x-2}{3x+1}):(x+1-\frac{2x^2+x+2}{3x+1}); \; \; x=7,(3)$

\texttt{ratsimp((x\^{}2+2*x+(11*x-2)/(3*x+1))/(x+1-(2*x\^{}2+x+2)/(3*x+1)));}\\
%%%%%%%%%%%%%%%
Ответ: $\frac{-2+13x+7x^2+3x^3}{x^2+3x-1}$
%%%%%%%%%%%%%%%
\\
Вычислим при $x = 3:$

\texttt{ratsubst(3, x, (-2+13*x+7*x\^{}2+3*x\^{}3)/(x\^{}2+3*x-1));}\\
%%%%%%%%%%%%%%%
Ответ: $\frac{181}{17}$
%%%%%%%%%%%%%%%
\\
Вычислим при $x = 7:$

\texttt{ratsubst(7, x, (-2+13*x+7*x\^{}2+3*x\^{}3)/(x\^{}2+3*x-1));}\\
%%%%%%%%%%%%%%%
Ответ: $\frac{487}{23}$
%%%%%%%%%%%%%%%
\\

№ 1.028\\
Упростить выражение:

$ \frac{\left(\sqrt[3]{\sqrt{(r^2+4)\sqrt{1+\frac{4}{r^2}}}}-\sqrt[3]{\sqrt{(r^2-4)\sqrt{1-\frac{4}{r^2}}}}\right)^2}{r^2-\sqrt{r^4-16}}$


\texttt{ratsimp((((r\^{}2+4)*sqrt(1+4/r\^{}2))\^{}(1/3)\\-((r\^{}2-4)*sqrt(1-4/r\^{}2))\^{}(1/3))/(r\^{}2-sqrt(r\^{}4-16)));}\\

Ответ:
$-\frac{\sqrt{r^2+4}-\sqrt{r^2-4}}{\sqrt{r^4-16}|r|^{\frac{1}{3}}-r^2|r|^{\frac{1}{3}}}$
%%%%%%%%%%%%%%%
\\

№ 1.043\\
Упростить выражение:
$$ \frac{x-1}{x+x^{0.5}+1}:\frac{x^{0.5}+1}{x^{1.5}-1}+\frac{2}{x^{-0.5}}$$
\texttt{ratsimp(((x-1)/(x+x\^{}(1/2)+1))/((x\^{}(1/2)+1)/(x\^{}(3/2)-1))+2/(x\^{}(-1/2)));}\\
Ответ: 
$x+1$

%%%%%%%%%%%%%%%

\newpage
\section{Алгебра матриц}
\begin{itemize}
\item[1.] Найти произведение матриц A и B. Произведение двух матриц в \emph{Maxima} обозначается символом "."\\
$$
A=\begin{pmatrix}
1 & 2 & 2 & -1\\
2 & 3 & 4 & 5\\
1 & 3 & 2 & 5\\
3 & 2 & 4 & -3
\end{pmatrix}; \; \; \;
B=\begin{pmatrix}
-5 & 2 & -1\\
-1 & 7 & 3\\
-2 & 4 & -3\\
1 & 3 & 2
\end{pmatrix}$$\\
\texttt{A: matrix([1,2,2,-1],[2,3,4,5],[1,3,2,5],[3,2,4,-3]);}\\
\texttt{B: matrix([-5,2,-1],[-1,7,3],[-2,4,-3],[1,3,2]);}
%%%%%%%%%%%%%%%
\\
$$A.B = 
\begin{pmatrix}
-12 & 21 & -3\\
-16 & 56 & 5\\
-7 & 46 & 12\\
-27 & 27 & -15 \\
\end{pmatrix}$$
%%%%%%%%%%%%%%%
\item[2.] Вычислить определитель матрицы A. Для подсчета определителя матрицы воспользуемся командой \textbf{determinant(A)}.\\
$$\begin{pmatrix}
1 & 2 & 3 & 4 & 5\\
2 & 3 & 7 & 10 & 13\\
3 & 5 & 11 & 16 & 21\\
2 & -7 & 7 & 7 & 2\\
1 & 4 & 5 & 3 & 10 \\
\end{pmatrix}$$\\
\texttt{determinant(matrix([1,2,3,4,5],[2,3,7,10,13],[3,5,11,16,21],}\\
\texttt{[2,-7,7,7,2],[1,4,5,3,10]));}\\
%%%%%%%%%%%%%%%
Ответ: 52



\item[3.] Дана матрица A. Найти матрицу \(A^{-1}\) и установить, что \(AA^{-1}=E\). Для начала обратим матрицу А, и запишем ее как матрица B. Чтобы обратить матрицу, существует команда \textbf{invert(A)}.\\
\texttt{A: matrix([17,10,4],[1,1,0],[2,-3,3]);}\\
$$A=\begin{pmatrix}
17 & 10 & 4 \\
1 & 1 & 0 \\
2 & -3 & 3 \\
\end{pmatrix}
$$\\

\texttt{B: invert(A);}\\
%%%%%%%%%%%%%%%
$$B=\begin{pmatrix}
3 & -42 & -4 \\
-3 & 43 & 4 \\
-5 & 71 & 7 \\
\end{pmatrix}$$\\
%%%%%%%%%%%%%%%
Покажем, что произведение матриц A и B есть единичная матрица:\\
$$A.B=\begin{pmatrix}
1 & 0 & 0\\
0 & 1 & 0\\
0 & 0 & 1
\end{pmatrix}$$\\
Получили единичную матрицу.
%%%%%%%%%%%%%%%
\end{itemize}

\newpage
\section{Функции математического анализа}
\begin{itemize}
\item[1.] Найти пределы выражений: № 746, № 761, № 776 [Из сборника задач по высшей математике, Минорский В. П.]. Для нахождения предела воспользуемся командой \textbf{limit(функция, переменная, точка)}.

№ 746\\
Найти предел:
$$\lim_{x \to \infty}\frac{2x^2-1}{3x^2-4x}$$
Запишем функцию:
$$F=\frac{2x^2-1}{3x^2-4x}$$

\texttt{F: (2*x\^{}2-1)/(3*x\^{}2-4*x);}\\
\texttt{limit(F, x, $\infty$)}; \\ 
Ответ: $\frac{2}{3}$

№ 761\\
Найти предел:
$$\lim_{x \to 7}\frac{2-\sqrt{x-3}}{x^2-49}$$
Запишем функцию:
$$F=\frac{2-\sqrt{x-3}}{x^2-49}$$

\texttt{F: (2-sqrt(x-3))/(x\^{}2-49);}\\
\texttt{limit(F, x, 7)}; \\ 
Ответ: $-\frac{1}{56}$

№ 776\\
Найти предел:
$$\lim_{x \to 0}\frac{2x\sin{x}}{\sec{x}-1}$$
Запишем функцию:
$$F=\frac{2x\sin{x}}{\sec{x}-1}$$

\texttt{F: (2*x*sin(x))/(sec(x)-1);}\\
\texttt{limit(F, x, 0)}; \\ 
Ответ: $4$

\item[2.] Вычислить производные: № 886, № 901 [Из сборника задач по высшей математике, Минорский В. П.]. Для вычисления производной воспользуемся командой \textbf{diff(y(x), x)}, где y(x) - функция, x - переменная, по которой дифференцируем.

№ 886\\
Найти производную функции:\\
1) Дана функция: $$ y = \frac{1}{(1+\cos{4x})^5} $$
\texttt{y(x):=1/(1+cos(4*x))\^{}5;}\\
\texttt{diff(y(x),x);}\\
Ответ: \[\frac{20\sin{(4x)}}{{{\left( 1+\cos{(4x)}\right) }^{6}}}\]

%%%%%%%%%%%%%%%

2) Дана функция: $$ s = \sqrt{\frac{t}{2} - sin{\frac{t}{2}}} $$
\texttt{s(t):=sqrt(t/2-sin(t/2))}\\
\texttt{diff(s(t),t);}\\
Ответ:
\[\frac{\frac{1}{2}-\frac{\cos{\left( \frac{t}{2}\right) }}{2}}{2\sqrt{\frac{t}{2}-\sin{\left( \frac{t}{2}\right) }}}\]
%%%%%%%%%%%%%%%
\item[3.] Найти интегралы: № 1547, № 1562, № 1577 [Из сборника задач по высшей математике, Минорский В. П.]. Для нахождения интеграла воспользуемся командой \textbf{integrate(y, x)}, где y - функция, x - переменная, по которой дифференцируем.

№ 1547\\
Найти интеграл:
$$\int\frac{\sin{x}dx}{b^2+\cos^2{x}}$$

\texttt{integrate(sin(x)/(b\^{}2+cos\^{}2(x)),x);}\\

Ответ: \[-\frac{\operatorname{atan}\left( \frac{\cos{(x)}}{b}\right) }{b}\]

%%%%%%%%%%%%%%%

№ 1562\\
Найти интеграл:
$$\int\frac{dx}{\sqrt{x+a}+\sqrt{x}}$$

\texttt{integrate(1/(sqrt(x+a)+sqrt(x)),x);}\\

Ответ: данный интеграл в системе wxMaxima не берётся.
Посчитаем через wolfram alpha: $$\frac{2(-x^{3/2}+a\sqrt{a+x}+x\sqrt{a+x})}{3a}$$

%%%%%%%%%%%%%%%

№ 1577\\
Найти интеграл:
$$\int e^{-\sqrt{x}}dx$$

\texttt{integrate(e\^{}sqrt(x),x);}\\

Ответ: \[-\frac{\left( 2+2\log{(e)}\,\sqrt{x}\right) \,{{\%{}e}^{-\log{(e)}\,\sqrt{x}}}}{{{\log{(e)}}^{2}}}\]

Ответ в системе wolfram: $-2e^{-\sqrt{x}}(\sqrt{x}+1)$

\item[4.] Вычислить площадь, ограниченную линиями: № 1637, № 1652, № 1667 [Из сборника задач по высшей математике, Минорский В. П.]. Для вычисления площади воспользуемся определенным интегралом.


\item[5.] Определить объем тела, образованного вращением фигуры, ограниченной линиями:\\

№ 1681 \\
$y = \sin{x}$(одной полуволной), $y=0$ вокруг оси $Ox$.

\end{itemize}

\section*{СПИСОК ИСПОЛЬЗОВАННЫХ ИСТОЧНИКОВ}
\addcontentsline{toc}{section}{СПИСОК ИСПОЛЬЗОВАННЫХ ИСТОЧНИКОВ}
\begin{itemize}
\item[1] Ильин В. А., Позняк Э. Г. Линейная алгебра. М. : Наука, 1984. 294с.
\item[2] Берсенев С. М., Иванов И. О. О вычислительных схемах метода регуляризации  // Журн. вычисл. мат. и мат. физ. 1984. Т. 24, № 9. 1402–1405c. 
\item[3] Фомин А. Е. О компонентах групп // Исследования по теории групп: сб. науч. тр. Свердловск, 1984. 136–148c.
\item[4] Хромов А. П. Конечномерные возмущения вольтерровых операторов: дис. ... д-ра физ.-мат. наук. Новосибирск, 1973. 242с.
\item[5] Губина Т. Н., Андропова Е.В. Решение дифференциальных уравнений в системе компьютерной математике Maxima: Учебное пособие. Елец. 2009. 99с.
\item[6] Стахин Н. А. Основы работы с системой аналитических вычислений Maxima: Учебное пособие. М. 2008. 86с.
\item[7] Виноградов И. М. Математическая энциклопедия. М.: Советская энциклопедия. 1977—1985.
\end{itemize}
\end{document}